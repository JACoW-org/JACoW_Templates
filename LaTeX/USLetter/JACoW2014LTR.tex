\documentclass[acus]{jacow}

\ifboolexpr{bool{xetex} or bool{luatex}}
 {}
 {\usepackage[utf8]{inputenc}}

\usepackage[USenglish]{babel}


\ifboolexpr{bool{jacowbiblatex}}%Nur wegen der Test nötig.
 {%
  \addbibresource{jacow-test.bib}
  \addbibresource{biblatex-examples.bib}
 }{}


%%
%%   Lengths for the spaces in the title
%%   \setlength\titleblockstartskip{..}  %before title, default 3pt
%%   \setlength\titleblockmiddleskip{..} %between title + author, default 1em
%%   \setlength\titleblockendskip{..}    %afterauthor, default 1em

\copyrightspace %default 1cm. Höher z.B. mit \copyrightspace[2cm]

%Für Tests
%\usepackage{eso-pic}
%\AddToShipoutPictureFG*{\AtTextLowerLeft{\textcolor{red}{COPYRIGHTSPACE}}}

\begin{document}


\title{preparation OF papers for \NoCaseChange{JACoW} conferences\thanks{Work supported by ...}}

\author{J. Poole, C. Petit-Jean-Genaz, CERN, Geneva, Switzerland\\
        C. Eyberger\thanks{cee@aps.anl.gov}, ANL, Argonne, IL 60439, USA\\
        V. RW Schaa, GSI,  Darmstadt, Germany}

\maketitle

%
\begin{abstract}
   Many conference series have adopted the same standards for electronic
   publication and have joined the Joint Accelerator Conference Website (JACoW)
   collaboration for the publication of their proceedings.
   This document describes the common requirements for the submission of papers
   to these conferences. Please consult individual conference
   information for page limits, method of electronic submission, etc.
   It is not intended that this should be a tutorial in word processing;
   the aim is to explain the particular requirements for electronic publication
   at these conference series.
\end{abstract}

\section{submission of papers}
Each\footnote{test footnote Work supported by blab blab blab blab blab blabl blab blab blab blab blab blabl blab blab blab blab blab blabl\ldots} author should submit the PDF (or PostScript) and all of the source files (text and figures),
to enable the paper to be reconstructed if there are processing difficulties.

\section{manuscripts}
Templates are provided for recommended software and authors are
advised to use them. Please consult the individual conference help pages if questions
arise.

\subsection{General Layout}

These instructions are a typical implementation of the
requirements. Manuscripts should have:
\begin{Itemize}
    \item  Either A4 (\SI{21.0}{cm}~$\times$~\SI{29.7}{cm}; \SI{8.27}{in}~$\times$~\SI{11.69}{in}) or US
           letter size (\SI{21.59}{cm}~$\times$~\SI{27.9}{cm}; \SI{8.5}{in}~$\times$~\SI{11}{in}) paper.
    \item  \textit{Single-spaced} text in two columns of \SI{82.5}{mm} (\SI{3.25}{in}) with \SI{5.3}{mm}
           (\SI{0.2}{in}) separation. Newer versions of Word (2007, 2010) have a default spacing
           of 1.5 lines; authors must change this to 1 line.
    \item  The text located within the margins specified in Table~\ref{l2ea4-t1}
           to facilitate electronic processing of the PDF file.
\end{Itemize}
\begin{table}[hbt]
   \centering
   \caption{Margin Specifications}
   \begin{tabular}{lcc}
       \toprule
       \textbf{Margin} & \textbf{A4 Paper}                      & \textbf{US Letter Paper} \\
       \midrule
           Top         & \SI{37}{mm} (\SI{1.46}{in})            & \SI{0.75}{in} (\SI{19}{mm})        \\
          Bottom       & \SI{19}{mm} (\SI{0.75}{in})            & \SI{0.75}{in} (\SI{19}{mm})        \\
           Left        & \SI{20}{mm} (\SI{0.79}{in})            & \SI{0.79}{in} (\SI{20}{mm})        \\
           Right       & \SI{20}{mm} (\SI{0.79}{in})            & \SI{1.02}{in} (\SI{26}{mm})        \\
       \bottomrule
   \end{tabular}
   \label{l2ea4-t1}
\end{table}

The layout of the text on the page is illustrated in
Fig.~\ref{l2ea4-f1}. Note that the paper's title and the author list should be
the width of the full page. Tables and figures may span the whole \SI{170}{mm} page width,
if desired (see Fig.~\ref{l2ea4-f2}), but they should be placed at
either the top or bottom of a page to ensure a proper flow of the text
(Word templates only: the text should flow from top to bottom in each column).

\begin{figure}[!htb]
   \centering
   \includegraphics*[width=65mm]{JACpic_mc}
   \caption{Layout of papers.}
   \label{l2ea4-f1}
\end{figure}

\subsection{Fonts}

In order to produce good Adobe Acrobat PDF files,
authors using a \LaTeX\ template are asked to use only Times (in roman (standard),
bold or italic) and symbols from  the standard set of fonts. In Word use only Symbol
and, depending on your platform, Times or Times New Roman fonts in standard, bold or
italic form.

\begin{figure*}[!tbh]
    \centering
    \includegraphics*[width=\textwidth]{JACpic3v2}

    \caption{Example of a full-width figure showing the JACoW Team at their annual
             meeting in 2012. The figure carries a multi-line caption which has to
             be justified, rather than centered.}
    \label{l2ea4-f2}
\end{figure*}

\subsection{Title and Author List}

The title should use \SI{14}{pt} bold uppercase letters and be centred on the page.
Individual letters may be lowercase to avoid misinterpretation (e.g., mW, MW).
To include a funding support statement, put an asterisk after the title and a
footnote at the bottom of the first column on page 1; in \LaTeX\ use
$\backslash$\texttt{thanks}.

The names of authors, their organisations/affiliations and mailing addresses
should be grouped by affiliation and listed in \SI{12}{pt} upper and lowercase letters.
The name of the submitting or primary author should be first, followed by
the co-authors, alphabetically by affiliation.


\subsection{Section Headings}

Section headings should not be numbered. They should
use  \SI{12}{pt}  bold  uppercase  letters  and  be  centred  in  the
column. All section headings should appear directly above
the text -- there should never be a column break between a heading and the
following paragraph.

\subsection{Subsection Headings}

Subsection  headings  should  not  be  numbered.
They should use \SI{12}{pt} italic letters and be left aligned in the column.
Subsection headings use \emph{T}itle \emph{C}ase (or \emph{I}nitial \emph{C}aps)
and should appear directly above the text -- there should never be a column break
between a heading and the following paragraph.

\subsubsection{Third-level Headings} use \LaTeX's \verb|\subsubsection|;
This is a new style for Word, where authors must bold text themselves.
Third-level headings should be used sparingly. See Table~\ref{style-tab} for its
style details.

\subsection{Paragraph Text}

Paragraphs should use \SI{10}{pt} font and be justified (touch each side) in
the column. The beginning of each paragraph should be indented
approximately \SI{3}{mm} (\SI{0.13}{in}). The last line of a paragraph should not be
printed by itself at the beginning of a column nor should the first line of
a paragraph be printed by itself at the end of a column.

\subsection{Figures, Tables and Equations}

Place figures and tables as close to the place of their mention as
possible. Lettering in figures and tables should be large enough to
reproduce clearly. Use of non-approved fonts in figures often leads to
problems when the files are processed. \LaTeX\ users -- please be sure to use
non-bitmapped versions of Computer Modern fonts in equations (Type\,1 PostScript
or OpenType fonts are required, and their use is described in the JACoW help
pages~\cite{jacow-help}).

All figures and tables must be given sequential numbers (1, 2, 3, etc.) and
have a caption (\SI{10}{pt} font) placed below the figure or above the table being described.
Captions that are one line should be centred in the column, while captions
that span more than one line should be justified. The \LaTeX\ template uses the \enquote*{booktabs}
package to format the tables.

A simple way to introduce figures into a Word document is to place them inside a table which has no borders. This is done in Word as follows:

\begin{Itemize}
\item	Insert a continuous section break.
\item	Insert two empty lines (makes later editing easier).
\item	Insert another continuous section break.
\item	Click between the two section breaks and Format $\rightarrow$ columns $\rightarrow$ Single.
\item	Table $\rightarrow$ Insert single column, two row table.
\item	Paste the figure in the first row and adjust the size as appropriate.
\item	Paste/Type the caption in the second row and apply figure caption style.
\item	Table $\rightarrow$ Table properties $\rightarrow$ Borders and shading $\rightarrow$ None.
\item	Table $\rightarrow$ Table properties $\rightarrow$ Alignment $\rightarrow$ Center.
\item	Table $\rightarrow$ Table properties $\rightarrow$ Text wrapping $\rightarrow$ None.
\item	Remove the blank lines from in and around the table.
\item	If necessary play with the cell spacing and other parameters to improve appearance.
\end{Itemize}

If a displayed equation needs a number (meaning it will be referenced), place it flush with the right
margin of the column (see Eq.~\ref{eq:units}). The equation itself should be indented (centered, if possible).
Units should be written
using the roman (standard) font, not the italic font.

\begin{equation}\label{eq:units}
    C_B=\frac{q^3}{3\epsilon_{0} mc}=\SI{3.54}{\micro eV/T}
\end{equation}

\subsection{References}

All bibliographical and web references should be numbered and listed at the
end of the paper in a section called \enquote{References}. When referring to a
reference in the text, place the corresponding reference number in square
brackets~\cite{exampl-ref}. The references in the text should appear in numerical
order.

A URL may be added as part of a reference, but
its hyperlink should NOT be added. Multiple citations should appear in
%the same square bracket~\cite{accelconf-ref, exampl-ref2, exampl-ref3} or
the same square bracket~\cite{exampl-ref2, exampl-ref3} or
%with ranges like~\cite{accelconf-ref}--\cite{exampl-ref3} or \cite{accelconf-ref, jacow-help, exampl-ref, exampl-ref2, exampl-ref3}.
with ranges like~\cite{exampl-ref2}--\cite{exampl-ref3} or \cite{exampl-ref2, jacow-help, exampl-ref, exampl-ref2, exampl-ref3}.

\subsection{Footnotes}

Footnotes on the title and author lines may be used for acknowledgements,
affiliations and e-mail addresses. A nonnumeric sequence of characters (*,
\dag, \ddag, \S) should be used. All other notes should use the normal numeric
sequencing and appear as footnotes\footnote{This text should appear
in the column where it was referenced.} in the same column where they are used.

Word users--do not use Word's footnote feature (\textbf{Insert, Footnote})
to insert footnotes as this will create formatting problems. Instead, insert
footnotes manually in a text box at the bottom of the first column with a
line at the top of the text box to separate the footnotes from the rest of
the paper's text.  The easiest way to do this is to copy the text box from
the JACoW template and paste it into your own document.

\subsection{Acronyms}

Acronyms should be defined the first time they appear.

\section{styles}

Table~\ref{style-tab} summarizes the fonts and spacings used in the styles of
a JACoW template (these are implemented in the \LaTeX\ class file).
\begin{table}[h!t]
    \setlength\tabcolsep{4pt}
    \caption{Summary of Styles}
    \label{style-tab}
    \begin{tabular}{@{}llcc@{}}
        \toprule
        \textbf{Style} & \textbf{Font}               & \textbf{Space}  & \textbf{Space} \\
                       &                             & \textbf{Before} & \textbf{After} \\
        \midrule
         Heading 1,    & \SI{14}{pt}                 & \SI{0}{pt}      & \SI{3}{pt}  \\
          PaperTitle   & Upper case except for       &                 &      \\
                       & required lower case letters &                 &      \\   %corrected 080515 vrws requred
                       & Bold                        &                 &      \\
         \midrule
          Author list  & \SI{12}{pt}                 & \SI{9}{pt}      & \SI{12}{pt} \\
                       & Upper and Lower case        &                 &      \\
         \midrule
         Heading 2,    & \SI{12}{pt}                 & \SI{9}{pt}      & \SI{3}{pt}  \\
         Section       & Uppercase                   &                 &      \\
         Heading       & bold                        &                 &      \\
        \midrule
         Heading 3,    & \SI{12}{pt}                 & \SI{6}{pt}      & \SI{3}{pt}  \\
         Subsection    & Initial Caps                &                 &      \\
         Heading       & Italic                      &                 &      \\
        \midrule
         Third-level   & \SI{10}{pt}                 & \SI{6}{pt}           & \SI{0}{pt}  \\
         Heading       & Initial Caps                &                 &      \\
                       & Bold                        &                 &      \\
        \midrule
         Figure        & \SI{10}{pt}                 & \SI{3}{pt}           & \SI{6}{pt}  \\
         Captions      &                             &                 &      \\
        \midrule
         Table         & \SI{10}{pt}                      & \SI{3}{pt}           & \SI{3}{pt}  \\
         Captions      &                             &                 &      \\
        \midrule
         Equations     & \SI{10}{pt} base font            & \SI{12}{pt}          & \SI{12}{pt} \\
        \midrule
         References    & \SI{9/10}{pt}, justified with  \SI{0.25}{in} &      &  \\
                       & hanging indent, reference   & $\ge$\SI{0}{pt} & $\ge$\SI{0}{pt}  \\
                       & number right aligned     &                 &        \\
        \bottomrule
    \end{tabular}
\end{table}

\section{page numbers}

\textbf{DO NOT have any page numbers}. They will be added
when the final proceedings are produced.

\section{templates}

Templates and examples can be retrieved through Web
browsers such as Firefox, Chrome and Internet Explorer by saving to disk.

Template documents for the recommended word processing software are
available from the JACoW Website (\url{http://JACoW.org}) and exist for
\LaTeX, Microsoft Word (Mac and PC) and OpenOffice for US letter and A4 paper sizes.

Use the correct templates for  your paper size and version of Word.
Do not transport Microsoft Word documents across platforms, e.g.
Mac~$\leftrightarrow$~PC. When saving a Word 2010 file (PC), be sure
to click `Embed fonts' in the Save options. Fonts are embedded by default
when printing to PDF (or Postscript) on Mac OSX.

Please see the information and help files for authors on the JACoW.org web site
for instructions  on  how to install templates in your Microsoft templates folder.

%\cite{gillies,doody,bertram,baez/article,angenendt}


\section{checklist for electronic publication}

\begin{Itemize}
    \item  Use only Times or Times New Roman (standard, bold or italic) and Symbol
           fonts for text -- \SI{10}{pt} minimum except references which can be \SI{9}{pt} or \SI{10}{pt}.
    \item  Figures should use Times or Times New Roman (standard, bold or italic) and
           Symbol fonts when possible~-- \SI{6}{pt} minimum.
    \item  Check that citations to references appear in sequential order and
           that all references are cited~\cite{exampl-ref11}.
    \item  Check that the PDF (or PostScript) file prints correctly.
    \item  Check that there are no page numbers.
    \item  Check that the margins on the printed version are within \SI{\pm1}{mm}
           of the specification.
    \item  \LaTeX\ users can check their margins by invoking the
           \texttt{boxit} option.
\end{Itemize}

\section{Conclusion}

Any conclusions should be in a separate section directly preceding
the \enquote{Acknowledgment}, \enquote{Appendix} and \enquote{References} sections in this
order.

\section{acknowledgment}
Any acknowledgment should be in a separate section directly preceding
the \enquote{References} or an \enquote{Appendix} section.

\section{appendix}
Any appendix should be in a separate section directly preceding
the \enquote{References} after an \enquote{Acknowledgment} or \enquote{Conclusion} section
in this order.


%\newpage
%\printbibliography

%\end{document}

%Falls ohne biblatex geht immer der manuelle Weg:
\newpage
\raggedend 

\begin{thebibliography}{99}   % Use for  1-9  references
%\begin{thebibliography}{99} % Use for 10-99 references

%\bibitem{accelconf-ref}
%	C. Petit-Jean-Genaz and J. Poole,
%	``JACoW, A service to the Accelerator Community,''
%	EPAC'04, Lucerne, July 2004, THZCH03,  p.~249,
%	\url{http://www.JACoW.org/e04/papers/THZCH03.PDF} \hfill\textcolor{red}{\{no period after URL\}}

\bibitem{jacow-help}
	A. Name and D. Person,
	Phys. Rev. Lett. 25 (1997) 56.

\bibitem{exampl-ref}
	A.N. Other,
	``A Very Interesting Paper,''
	EPAC'96, Sitges, June 1996, MOPCH31, p. 7984 (1996),
	\url{http://www.JACoW.org}  \hfill\textcolor{red}{\{no period after URL\}}

\bibitem{exampl-ref2}
	F.E.~Black et al.,
	``This is a Very Interesting Book",
	(New York: Knopf, 2007), 52.

\bibitem{exampl-ref3}
	G.B.~Smith et al.,
	``Title of Paper,'' MOXAP07, IPAC'13,
	to be published.\newline
	\hspace*{-1.1em}\mbox{\vdots}

\addtocounter{enumi}{5}
\bibitem{exampl-ref11}
	B.~Gnats, A. Jones,
	Phys. Rev. ST Accel. Beams 1, 011502 (1998).

\end{thebibliography}



\end{document}
